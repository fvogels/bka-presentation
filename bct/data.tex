\subsubsection{Observable Variables}
\frame{\tableofcontents[currentsubsection]}

\begin{frame}
    \frametitle{Probleemstelling}
    \begin{itemize}
        \item TUI
        \item Scheiding modellaag en presentatielaag
        \item Domein bevat data
        \item UI componenten krijgen elk klein stuk van domeindata te zien
    \end{itemize}
\end{frame}

\begin{frame}
    \frametitle{Design}
    \begin{center}
        \begin{tikzpicture}
            \draw[ultra thick] (0,0) -- (8,0);
            \node[anchor=south west] at (0,0) {Presentatie};
            \node[anchor=north west] at (0,0) {Domein};

            \node[anchor=north,font=\tiny,rectangle split,rectangle split parts=2,draw,inner sep=2pt] (item) at (4, -0.5) {
                \textbf{Item}
                \nodepart{two}
                \begin{tabular}{ll}
                    \textbf{Description} & "Blue shirt" \\
                    \textbf{Price} & 5 \\
                    \textbf{Category} & "Clothing 3-6 mos (56-62)" \\
                    \textbf{Donate} & true \\
                \end{tabular}
            };

            \node[anchor=south] at (4,0.2) { \includegraphics[height=3cm]{images/presentation.png} };
        \end{tikzpicture}
    \end{center}
\end{frame}

\begin{frame}
    \frametitle{Design}
    \begin{center}
        \begin{tikzpicture}[type/.style={minimum width=2cm,minimum height=1cm,fill=white,drop shadow,draw,rectangle split,font={\tiny}},
                            class/.style={type,rectangle split parts=3},
                            interface/.style={type,rectangle split parts=3}]
            \node[interface] (value) {
                \parbox{2cm}{
                    \centering
                    \textit{\textless\!\!\textless interface\textgreater\!\!\textgreater}\\
                    \textbf{Value[T any]}
                }
                \nodepart{three}
                \parbox{2cm}{
                    + Get():\ T
                }
            };

            \node[class,anchor=north east] (variable) at ($ (value.south west) + (-1,-1) $) {
                \textbf{Variable[T any]}
                \nodepart{two}
                \parbox{2cm}{
                    - value: T
                }
                \nodepart{three}
                \parbox{2cm}{
                    + Get():\ T \\
                    + Set(T)
                }
            };

            \node[class,anchor=north] (constant) at ($ (value.south) + (0,-1) $) {
                \textbf{Constant[T any]}
                \nodepart{two}
                \parbox{2cm}{
                    - value:\ T
                }
                \nodepart{three}
                \parbox{2cm}{
                    + Get():\ T
                }
            };

            \node[class,anchor=north west] (mappedvalue) at ($ (value.south east) + (1,-1) $) {
                \textbf{MappedValue[T, R any]}
                \nodepart{two}
                \parbox{2.25cm}{
                    - argument:\ Value[T] \\
                    - transformer:\ func(T) R
                }
                \nodepart{three}
                \parbox{2cm}{
                    + Get():\ T
                }
            };

            \draw[-{Triangle[open]},thick,dashed] (variable.north) |- (value.west);
            \draw[-{Triangle[open]},thick,dashed] (constant.north) -- (value.south);
            \draw[-{Triangle[open]},thick,dashed] (mappedvalue.north) |- (value.east);
        \end{tikzpicture}
    \end{center}
\end{frame}

\begin{frame}
    \frametitle{Design}
    \begin{center}
        \lstinputlisting[language=Go]{bct/valuemaps.go}
    \end{center}
\end{frame}

\begin{frame}
    \frametitle{Caching}
    \begin{itemize}
        \item Sommige transformaties zijn rekenintensief
        \item Caching
        \item Cache moet kunnen detecteren of herberekening nodig is
        \item Go biedt geen universele \texttt{equals} methode aan
        \item We werken daarom met versienummers
        \item Alternatief: elke value voorzien van een \texttt{func(T, T) bool}
    \end{itemize}
\end{frame}

\begin{frame}
    \frametitle{Design}
    \begin{center}
        \begin{tikzpicture}[type/.style={minimum width=2cm,minimum height=1cm,fill=white,drop shadow,draw,rectangle split,font={\tiny}},
                            class/.style={type,rectangle split parts=3},
                            interface/.style={type,rectangle split parts=3}]
            \node[interface] (value) {
                \parbox{2cm}{
                    \centering
                    \textit{\textless\!\!\textless interface\textgreater\!\!\textgreater}\\
                    \textbf{Value[T any]}
                }
                \nodepart{three}
                \parbox{2cm}{
                    + Get():\ T \\
                    \alert{+ Version():\ uint}
                }
            };

            \node[class,anchor=north east] (variable) at ($ (value.south west) + (-1,-1) $) {
                \textbf{Variable[T any]}
                \nodepart{two}
                \parbox{2cm}{
                    - value: T \\
                    \alert{- version: uint}
                }
                \nodepart{three}
                \parbox{2cm}{
                    + Get():\ T \\
                    + Set(T) \\
                    \alert{+ Version():\ uint}
                }
            };

            \node[class,anchor=north] (constant) at ($ (value.south) + (0,-1) $) {
                \textbf{Constant[T any]}
                \nodepart{two}
                \parbox{2cm}{
                    - value:\ T
                }
                \nodepart{three}
                \parbox{2cm}{
                    + Get():\ T \\
                    \alert{+ Version():\ uint}
                }
            };

            \node[class,anchor=north west] (cache) at ($ (value.south east) + (1,-1) $) {
                \textbf{Cache[T any]}
                \nodepart{two}
                \parbox{2.25cm}{
                    - value:\ Value[T] \\
                    - cached:\ T \\
                    - cachedVersion:\ uint
                }
                \nodepart{three}
                \parbox{2cm}{
                    + Get():\ T \\
                    + Version():\ uint
                }
            };

            \draw[-{Triangle[open]},thick,dashed] (variable.north) |- (value.west);
            \draw[-{Triangle[open]},thick,dashed] (constant.north) -- (value.south);
            \draw[-{Triangle[open]},thick,dashed] (mappedvalue.north) |- (value.east);
        \end{tikzpicture}
    \end{center}
\end{frame}

\begin{frame}
    \frametitle{Design}
    \begin{itemize}
        \item UI controls willen kunnen detecteren of waarde ge\"updatet werd sinds laatste rendering
        \item \texttt{View[T]} houdt bij welke versie laatst opgevraagd werd
    \end{itemize}
    \begin{center}
        \begin{tikzpicture}[type/.style={minimum width=2cm,minimum height=1cm,fill=white,drop shadow,draw,rectangle split,font={\tiny}},
                            class/.style={type,rectangle split parts=3},
                            interface/.style={type,rectangle split parts=3}]
            \node[interface] (value) {
                \parbox{2cm}{
                    \centering
                    \textit{\textless\!\!\textless interface\textgreater\!\!\textgreater}\\
                    \textbf{Value[T any]}
                }
                \nodepart{three}
                \parbox{2cm}{
                    + Get():\ T \\
                    \alert{+ Version():\ uint}
                }
            };

            \node[class,anchor=north] (view) at ($ (value.south) + (0,-1) $) {
                \textbf{View[T any]}
                \nodepart{two}
                \parbox{2.5cm}{
                    - value:\ Value[T] \\
                    - versionLastChecked:\ uint
                }
                \nodepart{three}
                \parbox{2.5cm}{
                    + Get():\ T \\
                    + Version():\ uint \\
                    + Updated():\ bool
                }
            };

            \draw[-{Triangle[open]},thick,dashed] (view.north) -- (value.south);
        \end{tikzpicture}
    \end{center}
\end{frame}

\begin{frame}
    \frametitle{Design}
    \lstinputlisting[language=Go]{bct/valueexample.go}
\end{frame}

\begin{frame}
    \frametitle{Design}
    \begin{itemize}
        \item Oorspronkelijk werkte TUI met observeerbare variabelen
        \item Observers kregen echter inconsistente data te zien
        \item Manier nodig om observer-updates te groeperen
        \item Laziness werd hiervoor ingevoerd
    \end{itemize}
\end{frame}
