\subsubsection{Observable Variables}
\frame{\tableofcontents[currentsubsection]}

\begin{frame}
    \frametitle{Probleemstelling}
    \begin{itemize}
        \item TUI
        \item Scheiding modellaag en presentatielaag
        \item Domein bevat data
        \item Presentatielaag krijgt \texttt{View}s, mogelijk transformaties van domeinwaardes
        \item Presentatielaag wil weten of waarde gewijzigd is
        \item Geen universele \texttt{equals} beschikbaar
    \end{itemize}
\end{frame}

\begin{frame}
    \frametitle{Design}
    \begin{center}
        \begin{tikzpicture}
            \draw[ultra thick] (0,0) -- (8,0);
            \node[anchor=south west] at (0,0) {Presentatie};
            \node[anchor=north west] at (0,0) {Domein};

            \node[anchor=north,font=\tiny,rectangle split,rectangle split parts=2,draw,inner sep=2pt] (item) at (4, -0.5) {
                \textbf{Item}
                \nodepart{two}
                \begin{tabular}{ll}
                    \textbf{Description} & "Blue shirt" \\
                    \textbf{Price} & 5 \\
                    \textbf{Category} & "Clothing 3-6 mos (56-62)" \\
                    \textbf{Donate} & true \\
                \end{tabular}
            };

            \node[anchor=south] at (4,0.2) { \includegraphics[height=3cm]{images/presentation.png} };
        \end{tikzpicture}
    \end{center}
\end{frame}

\begin{frame}
    \frametitle{Design}
    \begin{center}
        \begin{tikzpicture}[type/.style={minimum width=2cm,minimum height=1cm,fill=white,drop shadow,draw,rectangle split,font={\tiny}},
                            class/.style={type,rectangle split parts=3},
                            interface/.style={type,rectangle split parts=3}]
            \node[interface] (value) {
                \parbox{2cm}{
                    \centering
                    \textit{\textless\!\!\textless interface\textgreater\!\!\textgreater}\\
                    \textbf{Value[T any]}
                }
                \nodepart{three}
                \parbox{2cm}{
                    Get():\ T \\
                    Version():\ int
                }
            };
        \end{tikzpicture}
    \end{center}
\end{frame}

\begin{frame}
    \frametitle{Design}
    \begin{center}
        \lstinputlisting[language=Go]{bct/valuemaps.go}
    \end{center}
\end{frame}
